\documentclass[11pt]{article}

% Packages
\usepackage[utf8]{inputenc}
\usepackage{graphicx}
\usepackage{amsmath}
\usepackage{amssymb}
\usepackage{booktabs}
\usepackage{hyperref}
\usepackage{natbib}
\usepackage[margin=1in]{geometry}
\usepackage{caption}
\usepackage{subcaption}

% Title and authors
\title{Transfer Learning for Multi-Fidelity Surrogate Modeling of Delta Salinity Under Extended Drought and Climate Scenarios}

\author{
  Author One\textsuperscript{1},
  Author Two\textsuperscript{2},
  Author Three\textsuperscript{1}
  \\[0.5em]
  \small \textsuperscript{1}California Department of Water Resources, Delta Modeling Section \\
  \small \textsuperscript{2}Resource Management Associates, Inc.
}

\date{\today}

\begin{document}

\maketitle

\begin{abstract}
% TODO: Write abstract
\end{abstract}

% =============================================================================
\section{Introduction}
\label{sec:introduction}
% TODO: Background on Delta salinity management
% TODO: Challenge of extended drought and climate change
% TODO: Need for integrated modeling across scales
% TODO: Introduce transfer learning approach

% =============================================================================
\section{Background}
\label{sec:background}

\subsection{Sacramento-San Joaquin Delta System}
% TODO: Delta geography and water supply importance
% TODO: Salinity intrusion mechanisms
% TODO: D-1641 requirements and water cost concept

\subsection{Modeling Hierarchy}
% TODO: DSM2 (1D)
% TODO: SCHISM (3D)
% TODO: RMA Bay-Delta Model (2D/1D)
% TODO: CalSim operations model

\subsection{Surrogate Models for CalSim}
% TODO: Current ANN approach
% TODO: Limitations under novel scenarios

% =============================================================================
\section{Methods}
\label{sec:methods}

\subsection{Transfer Learning Framework}
% TODO: Staged training approach
% TODO: Contrastive loss formulation
% TODO: Multi-scenario architecture

\subsection{Model Architecture}
% TODO: GRU-based neural network
% TODO: Trunk and head structure
% TODO: Weight initialization strategy

\subsection{Training Pipeline}
% TODO: Data preprocessing
% TODO: Antecedent input creation
% TODO: Cross-validation strategy
% TODO: Loss functions and scaling

\subsection{Scenario Definition}
% TODO: Sea level rise scenarios
% TODO: Landscape changes (Franks Tract, Eco-Restore, Suisun, Cache)
% TODO: Temporary barriers

% =============================================================================
\section{Results}
\label{sec:results}

\subsection{Transfer Learning Performance}
% TODO: DSM2 to SCHISM transfer
% TODO: Base to scenario transfer
% TODO: NSE and skill metrics

\subsection{Multi-Scenario Model Performance}
% TODO: Parity with single-scenario approach
% TODO: Computational efficiency gains

\subsection{Round-Trip Validation}
% TODO: CalSim simulation results
% TODO: Multi-dimensional model validation
% TODO: Water cost estimates

% =============================================================================
\section{Discussion}
\label{sec:discussion}
% TODO: Implications for drought management
% TODO: Limitations
% TODO: Future work

% =============================================================================
\section{Conclusions}
\label{sec:conclusions}
% TODO: Key findings
% TODO: Recommendations

% =============================================================================
\section*{Acknowledgments}
% TODO: Delta Stewardship Council funding
% TODO: Collaborators

% =============================================================================
\bibliographystyle{apalike}
\bibliography{references}

\end{document}
